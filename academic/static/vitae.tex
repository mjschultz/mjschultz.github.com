% Academic curriculum vitae
% Maintained by Michael Schultz

\documentclass[12pt]{article}

\usepackage[margin=1in]{geometry}

\usepackage{enumitem}
\setitemize{itemsep=0pt,parsep=1pt}

\usepackage{multirow}
\usepackage{url}
\usepackage{fancyhdr}
\pagestyle{fancyplain}

% Special style for plain pages
\fancypagestyle{plain}{
	\fancyhf{}
	\fancyhead[R]{\today}
	\fancyfoot[R]{\thepage}
}

% Style for other pages
\fancyhf{}
\fancyhead[L]{Michael J. Schultz}
\fancyhead[R]{\today}
\fancyfoot[R]{\thepage}

\newcommand{\spantab}[3]{\multicolumn{#1}{@{\extracolsep{\fill}}#2}{#3}}

\begin{document}
\thispagestyle{plain}
\begin{center}
	{\Large {\bf Michael J. Schultz}} \\
	Curriculum vitae
\end{center}

\noindent\begin{tabular*}{\textwidth}{@{\extracolsep{\fill}}llll}
	Address:     & 5592 Waterman Blvd., Unit \#B3
	                   & Birth date:  & February 3, 1985    \\
	             & Saint Louis, MO 63112-1867, USA
	                   & Birth place: & Arlington Heights, IL, USA \\
	Email: & mjschultz@gmail.com
	                   & Phone: & +1 630 399 2814                        \\
    Web: &
    \multicolumn{3}{l}{\url{http://www.beyond-syntax.com/academic/}} \\
\end{tabular*}

\section*{Research Interests}

{\bf Networking}: network monitoring; high-speed networking.
{\bf Systems}: transactional memory; concurrency; scalable computing; virtual
machines.
{\bf Programming Languages}: compilers and optimization; correctness.
{\bf Software Engineering}: design patterns; static program analysis.

\section*{Educational Background}

\begin{tabular*}{\textwidth}{@{\extracolsep{\fill}}lr}
	{\bf Washington University in Saint Louis} & Saint Louis, Missouri \\
	\spantab{2}{l}{Doctor of Philosophy, {\em Computer Science},
                   Fall 2009--present.}   \\
	\spantab{2}{l}{Advisor: Patrick Crowley}                         \\
	\\
	{\bf Marquette University} & Milwaukee, Wisconsin              \\
	\spantab{2}{l}{Master of Science, {\em Computer Science},
	               August 2009.}                             \\
	\spantab{2}{l}{Thesis topic: {\em Using Software Transactional 
	               Memory in Interrupt-Driven Systems.}}           \\
	\spantab{2}{l}{Advisor: Dennis Brylow}                         \\
	\\
	{\bf Marquette University} & Milwaukee, Wisconsin              \\
	\spantab{2}{l}{Bachelor of Science, {\em Computer Science} and
		           {\em Mathematics}, May 2007.}                   \\
	\spantab{2}{l}{Major GPA: 3.72/4.00; Overall GPA: 3.33/4.00.}  \\
\end{tabular*}

\section*{Research Activities}

\subsection*{Papers}
\begin{itemize}
    \item {\bf Michael J. Schultz} and Patrick Crowley. {\em Performance
    Analysis of Packet Capture Methods in a 10 Gbps Virtualized
    Environment.} In the {\em 21st International Conference on Computer
    Communications and Networks}, Munich, Germany, July/August 2012.

    \item {\bf Michael J. Schultz}, Ben Wun, and Patrick Crowley. {\em A
    Passive Network Appliance for Real-Time Network Monitoring.} In the
    {\em ACM/IEEE Symposium on Architectures for Networking and
    Communications Systems}, Brooklyn, New York, October 2011.
\end{itemize}

\subsection*{Thesis}
\begin{itemize}
    \item {\bf Michael J. Schultz}. {\em Using Software Transactional Memory in
    Interrupt-Driven Systems.}  Master's Thesis, Marquette University, August
    2009.
\end{itemize}

\subsection*{Posters}
\begin{itemize}
    \item Adam T. Koehler, {\bf Michael J. Schultz}, and Dennis Brylow. Forward
    Thinking Poster Session 2008: {\em Solid State Persistent Memory Devices for
    Embedded Xinu}; Milwaukee, WI.
    \item {\bf Michael J. Schultz} and Dennis Brylow.  LCTES 2008 Student Poster
    Session: {\em Adding Software Transactions to the Kernel}; Tucson, AZ.
    \item Aaron Gember, Paul Hinze, {\bf Michael J. Schultz}, and Dennis Brylow.
    Forward Thinking Poster Session 2007: {\em When Machines have
    Deadlines---Building a Real-Time Embedded Operating System}; Milwaukee, WI.
\end{itemize}

\subsection*{Grants}
\begin{itemize}
    \item {\bf Michael J. Schultz, PI}. {\em Securing Industrial Control
    Networks with Network Forecasting.} NSF SBIR Award Number: 1248147,
    Award Amount: \$180,000.
\end{itemize}

\section*{Professional History}

\begin{tabular*}{\textwidth}{@{\extracolsep{\fill}}lr}
	{\bf Senior Engineer}        & August 2012 -- present          \\
    \spantab{2}{l}{Observable Networks, Inc.}                      \\
    \spantab{2}{l}{Designed and developed algorithms to analyze network
                   meta-data for malicious behavior; designed a scalable
                   architecture to handle growth in customers and data
                   volume.}                                        \\
	\\
	{\bf Research Assistant}     & August 2009 -- present          \\
    \spantab{2}{l}{Department of Computer Science and Engineering,
                   Washington University in Saint Louis}           \\
	\spantab{2}{l}{Applied Research Laboratory}                    \\
	\\
	{\bf Research Assistant}     & May 2007 -- August 2009         \\
	\spantab{2}{l}{Department of Mathematics, Statistics, 
	               and Computer Science, Marquette University}     \\
	\spantab{2}{l}{Systems Laboratory}                             \\
	\\
	{\bf Developer and Director} & September 2005 -- May 2007      \\
	\spantab{2}{l}{College of Communication, Student Media Interactive, 
                   Marquette University}                           \\
    \spantab{2}{l}{Led a team of student developers and designers for
                   the web presence of the newspaper, journal, radio,
                   television, and advertising.}                   \\
\end{tabular*}

\section*{Projects and Services}

\subsection*{Projects}
\begin{itemize}
	\item Embedded Xinu 1.0---released July 2007; ANSI C compliant release
	of the Xinu operating system targeting the MIPS platform; updated to
	v1.5 September 2008 with inclusion of an Ethernet driver and extensive
	code cleanup.
	\item Microwulf---built Fall 2008; portable four-node beowulf cluster.
\end{itemize}

\subsection*{Committees}
\begin{itemize}
	\item MSCS Undergraduate Committee---2006--2007 academic year; advocate
	for MSCS undergraduate interests.
	\item Student Media Board---2006--2007 academic year; helped begin
	convergence of student media outlets (paper, television, radio, and
	journal); assisted in hiring of managers for the next academic year.
\end{itemize}

\subsection*{Extra-Curricular}
\begin{itemize}
	\item ACM Student Chapter---{\em secretary} 2008--present
	\item Marquette University Linux Users Group---{\em co-founder} 2006;
	{\em president} 2006--2007; {\em system administrator} 2007--present.
\end{itemize}

\section*{Conferences Attended}

\begin{description}
    \item[ICCCN 2012] 21st International Conference on Computer
    Communications and Networks; Munich, Germany, July/August 2012.

    \item[ANCS 2011] ACM/IEEE Symposium on Architectures for Networking and
    Communications Systems; Brooklyn, NY, October 2011.

    \item[ANCS 2010] ACM/IEEE Symposium on Architectures for Networking and
    Communications Systems; La Jolla, CA, October 2010.

    \item[Strangeloop 2010] Strangeloop 2010; Saint Louis, MO, October
    2010.

	\item[SIGCSE 2010] ACM Technical Symposium in Computer Science
	Education; Milwaukee, WI, March 2010.

	\item[SIGCSE 2009] ACM Technical Symposium in Computer Science
	Education; Chattanooga, TN, March 2009.

	\item[EMSOFT 2008] International Conference on Embedded Systems;
	co-located with {\bf CASES} International Conference on Compilers,
	Architecture, and Synthesis for Embedded Systems; co-located with {\bf
	CODES+ISSS} International Conference on Hardware-Software Codesign and
	System Synthesis; Atlanta, GA, October 2008.

	\item[PLDI 2008] ACM SIGPLAN Conference on Programming Language Design
	and Implementation; co-located with {\bf LCTES 2008} ACM SIGPLAN/SIGBED
	Conference on Languages, Compilers, and Tools for Embedded Systems;
	Tucson, AZ, June 2008.

	\item[SIGCSE 2008] ACM Technical Symposium in Computer Science
	Education; Portland, OR, March 2008.
\end{description}

\section*{Professional and Scholarly Societies}
\begin{itemize}
	\item Association for Computing Machinery (ACM)
	\begin{itemize}
		\item SIGOPS: Special Interest Group on Operating Systems
	\end{itemize}
	\item Institute of Electrical and Electronic Engineers (IEEE) -
	Computer Society
	\item Upsilon Pi Epsilon - Honor Society in Computer Science
	\item Pi Mu Epsilon - Honor Society in Mathematics
\end{itemize}

\end{document}
